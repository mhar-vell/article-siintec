%% abtex2-modelo-artigo.tex, v<VERSION> laurocesar
%% Copyright 2012-<COPYRIGHT_YEAR> by abnTeX2 group at http://www.abntex.net.br/ 
%%
%% This work may be distributed and/or modified under the
%% conditions of the LaTeX Project Public License, either version 1.3
%% of this license or (at your option) any later version.
%% The latest version of this license is in
%%   http://www.latex-project.org/lppl.txt
%% and version 1.3 or later is part of all distributions of LaTeX
%% version 2005/12/01 or later.
%%
%% This work has the LPPL maintenance status `maintained'.
%% 
%% The Current Maintainer of this work is the abnTeX2 team, led
%% by Lauro César Araujo. Further information are available on 
%% http://www.abntex.net.br/
%%
%% This work consists of the files abntex2-modelo-artigo.tex and
%% abntex2-modelo-references.bib
%%

% ------------------------------------------------------------------------
% ------------------------------------------------------------------------
% abnTeX2: Modelo de Artigo Acadêmico em conformidade com
% ABNT NBR 6022:2018: Informação e documentação - Artigo em publicação 
% periódica científica - Apresentação
% ------------------------------------------------------------------------
% ------------------------------------------------------------------------

\documentclass[
	% -- opções da classe memoir --
	article,			% indica que é um artigo acadêmico
	12pt,				% tamanho da fonte
	oneside,			% para impressão apenas no recto. Oposto a twoside
	a4paper,			% tamanho do papel. 
	% -- opções da classe abntex2 --
	%chapter=TITLE,		% títulos de capítulos convertidos em letras maiúsculas
	%section=TITLE,		% títulos de seções convertidos em letras maiúsculas
	%subsection=TITLE,	% títulos de subseções convertidos em letras maiúsculas
	%subsubsection=TITLE % títulos de subsubseções convertidos em letras maiúsculas
	% -- opções do pacote babel --
	english,			% idioma adicional para hifenização
	brazil,				% o último idioma é o principal do documento
	sumario=tradicional
	]{abntex2}


% ---
% PACOTES
% ---

% ---
% Pacotes fundamentais 
% ---
\usepackage{lmodern}			% Usa a fonte Latin Modern
\usepackage[T1]{fontenc}		% Selecao de codigos de fonte.
\usepackage[utf8]{inputenc}		% Codificacao do documento (conversão automática dos acentos)
\usepackage{indentfirst}		% Indenta o primeiro parágrafo de cada seção.
\usepackage{nomencl} 			% Lista de simbolos
\usepackage{color}				% Controle das cores
\usepackage{graphicx}			% Inclusão de gráficos
\usepackage{microtype} 			% para melhorias de justificação
\usepackage{helvet}
\renewcommand{\familydefault}{\sfdefault}
\usepackage{parskip}
\usepackage{xcolor}
% ---
		
% ---
% Pacotes adicionais, usados apenas no âmbito do Modelo Canônico do abnteX2
% ---
\usepackage{lipsum}				% para geração de dummy text
% ---

% ---
% Pacotes de citações
% ---
\usepackage[brazilian,hyperpageref]{backref}	 % Paginas com as citações na bibl
\usepackage[num,abnt-emphasize=bf]{abntex2cite}	% Citações padrão ABNT
\citebrackets[]

% ---

% ---
% Configurações do pacote backref
% Usado sem a opção hyperpageref de backref
\renewcommand{\backrefpagesname}{Citado na(s) página(s):~}
% Texto padrão antes do número das páginas
\renewcommand{\backref}{}
% Define os textos da citação
\renewcommand*{\backrefalt}[4]{
	\ifcase #1 %
		Nenhuma citação no texto.%
	\or
		Citado na página #2.%
	\else
		Citado #1 vezes nas páginas #2.%
	\fi}%
% ---
%\nouppercaseheads
\makepagestyle{meuestilo}
  %%cabeçalhos
  \makeevenhead{meuestilo} %%pagina par
   %   {topo par à esquerda}
   %   {centro \thepage}
   %   {direita}
     {}{}{\vspace{-3em} \\ \footnotesize{VII INTERNATIONAL SYMPOSIUM ON INNOVATION AND TECHNOLOGY (SIINTEC)}\\ {\textit{One Planet, one Ocean and one Health. - 2021}\\ \vspace{\onelineskip}}}
  \makeoddhead{meuestilo} %%pagina ímpar ou com oneside
   %   {topo ímpar/oneside à esquerda}
   %   {centro\thepage}
   %   {direita}
      {}{}{\vspace{-3em} \\ \footnotesize{VII INTERNATIONAL SYMPOSIUM ON INNOVATION AND TECHNOLOGY (SIINTEC)}\\ {\textit{One Planet, one Ocean and one Health. - 2021}\\ \vspace{\onelineskip}}} 
  %\makeheadrule{meuestilo}{\textwidth}{\normalrulethickness} %linha
  %% rodapé
  \makeevenfoot{meuestilo}
      {ISSN: 2357-7592}{}{}
      % {rodapé par à esquerda} %%pagina par
      % {centro \thepage}
      % {direita} 
  \makeoddfoot{meuestilo} %%pagina ímpar ou com oneside
      {ISSN: 2357-7592}{}{}
      % {rodapé ímpar/onside à esquerda}
      % {centro \thepage}
      % {direita}

% --- Informações de dados para CAPA e FOLHA DE ROSTO ---
\titulo{TITLE ARIAL 14, BOLD AND UPPERCASE, JUSTIFIED, SIMPLE SPACE (\underline{in Portuguese}), WITH A MAXIMUM OF THREE LINES}

\tituloestrangeiro{TITLE ARIAL 14, BOLD AND UPPERCASE, JUSTIFIED, SINGLE SPACE (\underline{in English}), WITH A MAXIMUM OF THREE LINES}


\autor{
Nome Sobrenome\thanks{Instiutição A},  
Nome Sobrenome\thanks{Instituição B} 
}


% \local{Brasil}
% \data{2018, v<VERSION>}
% ---

% ---
% Configura\baselineskipr{blue}{RGB}{41,5,195}

% informações do PDF
\makeatletter

\hypersetup{
     	%pagebackref=true,
		pdftitle={\@title}, 
		pdfauthor={\@author},
    	pdfsubject={Modelo de artigo científico com abnTeX2},
	   pdfcreator={LaTeX with abnTeX2},
		pdfkeywords={abnt}{latex}{abntex}{abntex2}{atigo científico}, 
		colorlinks=true,       		% false: boxed links; true: colored links
    	linkcolor=blue,          	% color of internal links
    	citecolor=blue,        		% color of links to bibliography
    	filecolor=magenta,      		% color of file links
		urlcolor=blue,
		bookmarksdepth=4
}

\makeatother
% --- 

% ---
% compila o indice
% ---
\makeindex
% ---

% ---
% Altera as margens padrões
% ---
\setlrmarginsandblock{3cm}{3cm}{*}
\setulmarginsandblock{3cm}{3cm}{*}
\checkandfixthelayout
% ---

% --- 
% Espaçamentos entre linhas e parágrafos 
% --- 

% O tamanho do parágrafo é dado por:
\setlength{\parindent}{1.3cm}

% Controle do espaçamento entre um parágrafo e outro:
\setlength{\parskip}{0.2cm}  % tente também \onelineskip

% Espaçamento simples
\SingleSpacing


% ----
% Início do documento
% ----
\begin{document}

\pagestyle{meuestilo}
% Seleciona o idioma do documento (conforme pacotes do babel)
%\selectlanguage{english}
\selectlanguage{brazil}

% Retira espaço extra obsoleto entre as frases.
\frenchspacing 

%%usar o estilo criado na primeira página do artigo:
%\pretextual
%\pagestyle{meuestilo}
% ----------------------------------------------------------
% ELEMENTOS PRÉ-TEXTUAIS
% ----------------------------------------------------------

%---
%
% Se desejar escrever o artigo em duas colunas, descomente a linha abaixo
% e a linha com o texto ``FIM DE ARTIGO EM DUAS COLUNAS''.
% \twocolumn[    		% INICIO DE ARTIGO EM DUAS COLUNAS
%
%---

% página de titulo principal (obrigatório)
%\maketitle

%%usar o estilo criado na primeira página do artigo:
\pretextual
\pagestyle{meuestilo}

% titulo em outro idioma (opcional)


% resumo em inglês
% \renewcommand{\resumoname}{}
% \begin{resumoumacoluna}
%  \begin{otherlanguage*}{english}

   \vspace*{2pt}

   %\vspace{\onelineskip}
   \noindent
   \textbf{\large{TITLE ARIAL 14, BOLD AND UPPERCASE, JUSTIFIED, SINGLE SPACE (\underline{in English}), WITH A MAXIMUM OF THREE LINES}}
   
   \vspace{\onelineskip}
   \noindent
   \textit{
      Rafael Silva Teixeira$^1$, Second Author$^{a,b}$, Third Author$^b$ etc
   }
   \vspace{\onelineskip}

   \noindent
   $^{a, 1}$ \textit{Institute Name, Country}\\
   $^b$ \textit{Department, Institute Name, if any}

   \vspace*{1.2cm}
   \noindent
   \normalsize
   \textbf{Abstract:} The abstract must be written in Arial font, size 12, and within this area. The text should be justified. The abstract should include the objective, methodology, main results and conclusions. It should not exceed 10 lines.

   \vspace{\onelineskip}
   \noindent
   \textbf{Keywords}: must include 3 to 5 keywords, separated by semicolons.
%  \end{otherlanguage*}  
% \end{resumoumacoluna}


% resumo em português
%\begin{resumoumacoluna}
   \vspace*{1.5cm}
   \noindent
   \textbf{\large{TITLE ARIAL 14, BOLD AND UPPERCASE, JUSTIFIED, SIMPLE SPACE (\underline{in Portuguese}), WITH A MAXIMUM OF THREE LINES}}

   \vspace{\onelineskip}
   \noindent
   \normalsize
   \textbf{Resumo:} O resumo deve ser digitado em português e em fonte Arial tamanho 12, e dentro desta área. O texto deve ser justificado. O resumo deve conter o objetivo, metodologia, principais resultados e conclusões. Não deve ultrapassar 10 linhas.

   \vspace{\onelineskip}
   \noindent
   \textbf{Palavras-chave:} deve ter no mínimo de 3 a 5 palavras-chave, em português, separadas por ponto e vírgula.
%\end{resumoumacoluna}


% ]  				% FIM DE ARTIGO EM DUAS COLUNAS
% ---

% \begin{center}\smaller
% \textbf{Data de submissão e aprovação}: elemento obrigatório. Indicar dia, mês e ano

% \textbf{Identificação e disponibilidade}: elemento opcional. Pode ser indicado 
% o endereço eletrônico, DOI, suportes e outras informações relativas ao acesso.
% \end{center}

% ----------------------------------------------------------
% ELEMENTOS TEXTUAIS
% ----------------------------------------------------------
\textual
\pagestyle{meuestilo}


\newpage

% ----------------------------------------------------------
% Introdução
% ----------------------------------------------------------
\section{\textbf{INTRODUCTION (ARIAL 12, must start on the second page)}}

\textcolor{red}{THE ENTIRE MANUSCRIPT SHOULD BE WRITTEN IN ENGLISH.}

The heading of each section should be numbered with Arabic numerals and left-aligned, with uppercase and bold letters. A space should be added between the end of a section and the heading of the next section.

The font to be used is Arial, size 12, normal format. It is recommended to use THIS template to write the manuscript available for download at the event's website. All authors should meet the following recommendations to prepare their manuscript. The text should be justified. If any information is not included in this document, adopt the ABNT standard for the references.

The manuscripts must be submitted in digital format (Word and PDF) and full text. The manuscripts submitted to \textbf{VII SIINTEC} may be in English. The event is not responsible for works with poor quality of the graphs/figures is if they have not followed the guidelines provided herein. In addition, no revision of the texts will be made, and they will be the sole responsibility of their authors.

\textbf{\underline{The article page limit is between 6 (six) and 8 (eight)}}, considering text, tables, figures and photos. These pages should be in A4 format (210 x 297 mm), with lateral, top and bottom margins of 25 mm. There should be no text in the header and footer areas, only the template text.

References should be numbered in ascending order in the text and enclosed in square brackets (i.e., [1]).

Please contact the Organizing Committee of the event if you have any questions.


%//TODO parei aqui

\subsection{\textbf{Secondarry Section (arial 12)}}

Secondary headings should be left-aligned, bold, and the first letter of each word should be capitalized.

\subsubsection{\textbf{Tertiary Section (Arial 12)}}

Consecutive numbering should be used to systematize the content of the manuscript. It is necessary to use uppercase letters in the primary sections and bold in all sections.


\section{\textbf{METHODOLOGY (ARIAL 12)}}

In the Methodology, the type of study, site, population (in case of field research), period, technique and data analysis, as well as ethical standards followed (in case of human research) should be explained. In short, describe all the method(s) used to perform the study.
The figures can be colored and should be inserted in the body of the text, close to their citations in the text. The figures should be centered, without exceeding the size limited by the margins of the page.
Each figure must contain a title numbered in Arabic numerals. Titles should be written in Arial 11 font, and should be centered at the top of the figure.

//TODO figure

Tables should be centered (using the entire area between the margins). The tables can be colored or black and white. The title of the table should be centered and written in Arial 11 font. The measurement units corresponding to all variables should be clearly indicated using the International System (IS).


//TODO table

\section{\textbf{RESULTS AND DISCUSSION (ARIAL 12)}}

Equations should be written in italics with consecutive numbering in parentheses and should be right-justified. Equations with more than one line should be numbered on the last line, in parentheses and right-justified.



//TODO equation

If necessary, the list of notations and symbols used, as well as their units of measurement, should be listed before the references (in alphabetical order).



% Este documento e seu código-fonte são exemplos de referência de uso da classe
% \textsf{abntex2} e do pacote \textsf{abntex2cite}. O documento exemplifica a
% elaboração de publicação periódica científica impressa produzida conforme a ABNT
% NBR 6022:2018 \emph{Informação e documentação - Artigo em publicação periódica
% científica - Apresentação}.

% A expressão ``Modelo canônico'' é utilizada para indicar que \abnTeX\ não é
% modelo específico de nenhuma universidade ou instituição, mas que implementa tão
% somente os requisitos das normas da ABNT. Uma lista completa das normas
% observadas pelo \abnTeX\ é apresentada em \citeonline{abntex2classe}.

% Sinta-se convidado a participar do projeto \abnTeX! Acesse o site do projeto em
% \url{http://www.abntex.net.br/}. Também fique livre para conhecer,
% estudar, alterar e redistribuir o trabalho do \abnTeX, desde que os arquivos
% modificados tenham seus nomes alterados e que os créditos sejam dados aos
% autores originais, nos termos da ``The \LaTeX\ Project Public
% License''\footnote{\url{http://www.latex-project.org/lppl.txt}}.

% Encorajamos que sejam realizadas customizações específicas deste documento.
% Porém, recomendamos que ao invés de se alterar diretamente os arquivos do
% \abnTeX, distribua-se arquivos com as respectivas customizações. Isso permite
% que futuras versões do \abnTeX~não se tornem automaticamente incompatíveis com
% as customizações promovidas. Consulte \citeonline{abntex2-wiki-como-customizar}
% para mais informações.

% Este exemplo deve ser utilizado como complemento do manual da classe
% \textsf{abntex2} \cite{abntex2classe}, dos manuais do pacote
% \textsf{abntex2cite} \cite{abntex2cite,abntex2cite-alf} e do manual da classe
% \textsf{memoir} \cite{memoir}. Consulte o \citeonline{abntex2modelo} para obter
% exemplos e informações adicionais de uso de \abnTeX\ e de \LaTeX.

% % ----------------------------------------------------------
% % Seção de explicações
% % ----------------------------------------------------------
% \section{Exemplos de comandos}

% \subsection{Margens}

% A norma ABNT NBR 6022:2018 não estabelece uma margem específica a ser utilizada
% no artigo científico. Dessa maneira, caso deseje alterar as margens, utilize os
% comandos abaixo:

% \begin{verbatim}
%    \setlrmarginsandblock{3cm}{3cm}{*}
%    \setulmarginsandblock{3cm}{3cm}{*}
%    \checkandfixthelayout
% \end{verbatim}

% \subsection{Duas colunas}

% É comum que artigos científicos sejam escritos em duas colunas. Para isso,
% adicione a opção \texttt{twocolumn} à classe do documento, como no exemplo:

% \begin{verbatim}
%    \documentclass[article,11pt,oneside,a4paper,twocolumn]{abntex2}
% \end{verbatim}

% É possível indicar pontos do texto que se deseja manter em apenas uma coluna,
% geralmente o título e os resumos. Os resumos em única coluna em documentos com
% a opção \texttt{twocolumn} devem ser escritos no ambiente
% \texttt{resumoumacoluna}:

% \begin{verbatim}
%    \twocolumn[              % INICIO DE ARTIGO EM DUAS COLUNAS

%      \maketitle             % pagina de titulo

%      \renewcommand{\resumoname}{Nome do resumo}
%      \begin{resumoumacoluna}
%         Texto do resumo.
      
%         \vspace{\onelineskip}
 
%         \noindent
%         \textbf{Palavras-chave}: latex. abntex. editoração de texto.
%      \end{resumoumacoluna}
   
%    ]                        % FIM DE ARTIGO EM DUAS COLUNAS
% \end{verbatim}

% \subsection{Recuo do ambiente \texttt{citacao}}

% Na produção de artigos (opção \texttt{article}), pode ser útil alterar o recuo
% do ambiente \texttt{citacao}. Nesse caso, utilize o comando:

% \begin{verbatim}
%    \setlength{\ABNTEXcitacaorecuo}{1.8cm}
% \end{verbatim}

% Quando um documento é produzido com a opção \texttt{twocolumn}, a classe
% \textsf{abntex2} automaticamente altera o recuo padrão de 4 cm, definido pela
% ABNT NBR 10520:2002 seção 5.3, para 1.8 cm.

% \section{Cabeçalhos e rodapés customizados}

% Diferentes estilos de cabeçalhos e rodapés podem ser criados usando os
% recursos padrões do \textsf{memoir}.

% Um estilo próprio de cabeçalhos e rodapés pode ser diferente para páginas pares
% e ímpares. Observe que a diferenciação entre páginas pares e ímpares só é
% utilizada se a opção \texttt{twoside} da classe \textsf{abntex2} for utilizado.
% Caso contrário, apenas o cabeçalho padrão da página par (\emph{even}) é usado.

% Veja o exemplo abaixo cria um estilo chamado \texttt{meuestilo}. O código deve
% ser inserido no preâmbulo do documento.

% \begin{verbatim}
% %%criar um novo estilo de cabeçalhos e rodapés
% \makepagestyle{meuestilo}
%   %%cabeçalhos
%   \makeevenhead{meuestilo} %%pagina par
%      {topo par à esquerda}
%      {centro \thepage}
%      {direita}
%   \makeoddhead{meuestilo} %%pagina ímpar ou com oneside
%      {topo ímpar/oneside à esquerda}
%      {centro\thepage}
%      {direita}
%   \makeheadrule{meuestilo}{\textwidth}{\normalrulethickness} %linha
%   %% rodapé
%   \makeevenfoot{meuestilo}
%      {rodapé par à esquerda} %%pagina par
%      {centro \thepage}
%      {direita} 
%   \makeoddfoot{meuestilo} %%pagina ímpar ou com oneside
%      {rodapé ímpar/onside à esquerda}
%      {centro \thepage}
%      {direita}
% \end{verbatim}

% Para usar o estilo criado, use o comando abaixo imediatamente após um dos
% comandos de divisão do documento. Por exemplo:

% \begin{verbatim}
%    \begin{document}
%      %%usar o estilo criado na primeira página do artigo:
%      \pretextual
%      \pagestyle{meuestilo}
     
%      \maketitle
%      ...
     
%      %%usar o estilo criado nas páginas textuais
%      \textual
%      \pagestyle{meuestilo}
     
%      \chapter{Novo capítulo}
%      ...
%    \end{document}  
% \end{verbatim}
   
% Outras informações sobre cabeçalhos e rodapés estão disponíveis na seção 7.3 do
% manual do \textsf{memoir} \cite{memoir}.

% \section{Mais exemplos no Modelo Canônico de Trabalhos Acadêmicos}

% Este modelo de artigo é limitado em número de exemplos de comandos, pois são
% apresentados exclusivamente comandos diretamente relacionados com a produção de
% artigos.

% Para exemplos adicionais de \abnTeX\ e \LaTeX, como inclusão de figuras,
% fórmulas matemáticas, citações, e outros, consulte o documento
% \citeonline{abntex2modelo}.

% \section{Consulte o manual da classe \textsf{abntex2}}

% Consulte o manual da classe \textsf{abntex2} \cite{abntex2classe} para uma
% referência completa das macros e ambientes disponíveis.

% ---
% Finaliza a parte no bookmark do PDF, para que se inicie o bookmark na raiz
% ---
\bookmarksetup{startatroot}% 
% ---

% ---
% Conclusão
% ---
\section{\textbf{CONCLUSION (ARIAL 12)}}

In this document, the guidelines that should be followed by all authors for the publication of the manuscripts were described. As observed in these guidelines, the manuscript should be sent through the link to be provided at a later date.
The conclusion should not contain citations/references.


\subsection*{\textbf{Acknowledgments}}

The authors can write their acknowledgments before citing the references. When they thank the funding agencies, the number of the grant process or financial support must be provided in parentheses.

References must be numbered consecutively and listed. References must be numbered in the order they appear in the text and enclosed in square brackets (i.e., \cite{gomes1998novela}). In the example below, references \cite{gomes1998novela}, \cite{pinto2005biodiesel} and \cite{silva1998pena} refer, respectively, to books, articles published in journals and electronic documents. Use ABNT NBR 6023: 2002. When more than one reference is cited in the same paragraph, use the following citation model: \cite{gomes1998novela, pinto2005biodiesel}.

% ----------------------------------------------------------
% ELEMENTOS PÓS-TEXTUAIS
% ----------------------------------------------------------
\postextual

% ----------------------------------------------------------
% Referências bibliográficas
% ----------------------------------------------------------
\vspace*{1.5cm}

\bibliographystyle{abntex2-num}
\bibliography{abntex2-modelo-references}

% ----------------------------------------------------------
% Glossário
% ----------------------------------------------------------
%
% Há diversas soluções prontas para glossário em LaTeX. 
% Consulte o manual do abnTeX2 para obter sugestões.
%
%\glossary

% ----------------------------------------------------------
% Apêndices
% ----------------------------------------------------------

% ---
% Inicia os apêndices
% ---
\begin{apendicesenv}

   \newpage
% ----------------------------------------------------------
\section*{OTHER INFORMATION}
\begin{enumerate}[label=\alph*)]
   \item Manuscripts and information included therein are the responsibility of the authors and may not represent the opinion of \textbf{VII SIINTEC}.
   \item The authors accept that \textbf{VII SIINTEC} has full rights to the submitted manuscripts and may include them in the proceedings, print them and disclose them, without payment of any kind.
   \item Manuscripts will be evaluated by reviewers invited by the Scientific Committee of the Event. Only accepted manuscripts can be presented and published at the event.
\end{enumerate}

\raggedright For additional clarifications, contact:

Organizing Committee of the Event - \textit{siintec@fieb.org.br}

SENAI CIMATEC
% ----------------------------------------------------------

\end{apendicesenv}
% % ---

% % ----------------------------------------------------------
% % Anexos
% % ----------------------------------------------------------
% \cftinserthook{toc}{AAA}
% % ---
% % Inicia os anexos
% % ---
% %\anexos
% \begin{anexosenv}

% % ---
% \chapter{Cras non urna sed feugiat cum sociis natoque penatibus et magnis dis
% parturient montes nascetur ridiculus mus}
% % ---

% \lipsum[31]

% \end{anexosenv}

% ----------------------------------------------------------
% Agradecimentos
% ----------------------------------------------------------

% \section*{Agradecimentos}
% Texto sucinto aprovado pelo periódico em que será publicado. Último 
% elemento pós-textual.

\end{document}
